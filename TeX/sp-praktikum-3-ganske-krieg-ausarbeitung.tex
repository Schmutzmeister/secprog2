\documentclass[12pt]{article}

\parskip 1ex plus 0.4ex minus 0.4ex
\parindent0ex

\usepackage{enumerate}
\usepackage{prettyref}
\usepackage{pdfpages}
\usepackage{float}
\usepackage{dirtree}
\usepackage{xltxtra}
\usepackage{polyglossia}
\setdefaultlanguage[spelling=new]{german}
\usepackage{fontspec}
\usepackage{listings}
\usepackage{xcolor}
\usepackage{amsmath}
\usepackage{amsthm}
\usepackage{amstext}
\usepackage{amssymb}
\usepackage{graphicx}
\usepackage[colorlinks,
pdfpagelabels,
pdfstartview = FitH,
bookmarksopen = true,
bookmarksnumbered = true,
linkcolor = blue, %Farbe
plainpages = false,
hypertexnames = false,
citecolor = black,
xetex] {hyperref}
%\usepackage[style=alphabetic,backend=biber]{biblatex}
%\bibliography{bib}

\title{Praktikumsbericht: Secure C Coding}
\author{Marcus Ganske, 36603\\
		Lukas Krieg, 53506}
\date{\today}

\definecolor{light-gray}{gray}{0.95}

\lstset{
	language=c,
	breaklines,
	basicstyle=\ttfamily\small,
	backgroundcolor=\color{light-gray},
	keywordstyle=\color{blue},
	stringstyle=\color{olive},
	commentstyle=\color{gray}\ttfamily,
	numbers = left,
	numberstyle = \tiny,
	numberblanklines=true,
	stepnumber = 1,
	tabsize = 4,
	numbersep=10pt,
	xleftmargin=10pt,
}

\begin{document}
\maketitle
\vspace{+8cm}{
}
\includegraphics[width=12cm]{Hochschule-aalen.pdf}

\newpage
%Inhaltsverzeichnis
\renewcommand\contentsname{Inhaltsverzeichnis}
\tableofcontents
\newpage
%Den Code in das CMAKE Projekt von Karg einf\"ugen und in Code Ordner reinlegen

	
	\section{Einleitung}
    Dieser Bericht beschreibt das dritte Praktikum der Vorlesung Sicheres Programmieren. Die Richtlinien EXP37-C, INT30-C, MEM31-C und INT33-C der CERT Secure Coding Guidelines sollen untersucht werden. Au{\ss}erdem sollen Beispielimplementationen f\"ur das Versto{\ss}en und Befolgen der Richtlinen erstellt werden. Desweiteren sollen die Auswirkungen bei Nichtbeachtung der Richtlinen beschrieben werden.
	\begin{figure}[H]
			\dirtree{%Baumstruktur
				.1 Code/.
				.2 exp37-compliant.c.
				.2 exp37-noncompliant.c.
				.2 exp37-2-compliant.c.
				.2 exp37-2-noncompliant.c.
				.2 int30-compliant.c.
				.2 int30-noncompliant.c.
				.2 int30-2-compliant.c.
				.2 int30-2-noncompliant.c.
				.2 int33-compliant.c.
				.2 int33-noncompliant.c.
				.2 int33-2-compliant.c.
				.2 int33-2-noncompliant.c.
				.2 mem31-compliant.c.
				.2 mem31-noncompliant.c.
			}
	
		\caption{Aufbau des Ordners Code mit allen Dateien}
	\end{figure}	

\newpage
\section{EXP37-C: Call functions with the correct number and type of arguments}
\subsection{Beschreibung der Richtlinie}
Funktionen sollen nicht mit einer falschen Anzahl von Parametern oder mit Parametern eines falschen Typs aufgerufen werden.
Der C Standard beschreibt f\"unf verschiedene Situationen, in denen unvorhersehbares Verhalten auftreten kann. 
\begin{itemize}
    \item Ein Zeiger wird benutzt um eine Funktion aufzurufen, deren Typ nicht mit dem referenzierten Typ kompatibel ist.
    \item F\"ur einen Aufruf eine Funktion ohne Prototyp im scope. Die Anzahl der Argumente stimmt nicht mit der Anzahl der Parameter \"uberein.
    \item Wenn eine Funktion ohne Prototyp im scope aufgerufen wird, aber es einen Prototyp gibt, k\"onnen die Typen der Argumente des Prototypes vom Aufruf abweichen und inkompatibel sein.
    \item Wenn eine Funktion ohne Prototyp im scope aufgerufen wird und es keinen Prototyp gibt, k\"onnen die Typen der Parameter des Prototypes vom Aufruf abweichen und inkompatibel sein.
    \item Eine Funktion ist mit einem Typ definiert, der nicht mit dem der aufgerufenen Funktion \"ubereinstimmt.
\end{itemize}

Wird eine Funktion in Ansi C benutzt, die nicht deklariert wurde und von der kein Prototyp sichtbar ist, dann wird der Compiler die beim Funktionsaufruf benutzten Argumente als korrekten Aufruf der Funktion annehmen. Auch wenn Funktionen ordentlich deklariert sind und mit der falschen Anzahl oder falschem Typ von Argumenten aufgerufen werden, wird der Compiler bestenfalls nur Warnmeldungen ausgeben aber den Code trotzdem kompilieren.

\newpage

\subsection{Codebeispiele}
	In diesem Beispiel soll der Logarithmus zur Basis 2 einer komplexen Zahl bestimmt werden, was zu einem unbestimmten Verhalten f\"uhrt.\\
	In frt  
\subsubsection{nichtkonformer Code}
\lstinputlisting[linerange={5-150}]{../code/aufg3/CERT-C-Coding-Standard/src/Praktikum3/exp37-noncompliant.c}
\subsubsection{konformer Code}
\lstinputlisting[linerange={5-150}]{../code/aufg3/CERT-C-Coding-Standard/src/Praktikum3/exp37-compliant.c}
	In diesem Beispiel soll die Funktion strchr() aus der C-Standardbibliothek \"uber einen Funktionszeiger fp, dessen Prototyp mit falschen Datentypen deklariert wurde, aufgerufen werden.

\subsubsection{nichtkonformer Code}
\lstinputlisting{../code/aufg3/CERT-C-Coding-Standard/src/Praktikum3/exp37-2-noncompliant.c}
\subsubsection{konformer Code}
\lstinputlisting{../code/aufg3/CERT-C-Coding-Standard/src/Praktikum3/exp37-2-noncompliant.c}

\subsection{Auswirkung bei Nichtbeachtung}
Das Verhalten nach einem Funktionsaufruf mit falschen Argumenten ist nicht definiert. Die Funktion des Programms kann eingeschr\"ankt werden, beispielsweise k\"onnen Berechnungen falsche Ergebni{\ss}e liefern. Im schlimmsten Fall k\"onnen Sicherheitsl\"ucken auftreten. In den Shadow Utilities wurde eine Sicherheitsl\"ucke (VU\#312692) gefunden. Dabei hat bei der Utility useradd die Funktion Open() nicht die erwarteten Argumente \"ubergeben bekommen. Dies f\"uhrte dazu, da{\ss} zuf\"allige Berechtigungen f\"ur die neue Mailbox gesetzt wurden.

\newpage
\section{INT30-C: Ensure that unsigned integer operations do not wrap}
\subsection{Beschreibung der Richtlinie}
Das Wort wrap ist nicht zu verwechseln mit dem over bzw. underflow einere Integervariable. Bei einem wrap eines unsigned Integers kann die Vatriable ihren Wert nach einer Operation nichtmehr darstellen. 

Ist die Obergrenze der Variable erreicht, wird sie bei einer weiteren Erh\"ohung ihres Wertes der neue Wert (neuer Wert mod Obergrenze)sein.\\
Addition In Bin\"ardarstellung(32bit unsigned Integer):\\
11111111111111111111111111111111  // 4.294.967.295\\
+\\
00000000000000000000000000000001  // +1\\
=\\
00000000000000000000000000000000  // 0\\
Die Addition hat einen \"ubertrag des niederwertigsten Bits weitergetragen bis zum h\"ochstwertigsten Bit. Alle Bits sind nun 0 und der \"ubertrag ist verlorengegangen, da keine weiteren Bits vorhanden waren. In diesem Beispiel wurde ein wrap mit Addition herbeigef\"uhrt. \\
In diesem Beispiel wird ein wrap durch Subtraktion herbeigef\"uhrt:\\
00000000000000000000000000000000 // 0\\
-\\
00000000000000000000000000000001 // -1\\
=\\
11111111111111111111111111111111 // 4.294.967.295\\

Der Wert -1 kann von einem unsigned Integer nicht dargestellt werden. Das Ergebnis ist daher (-1 mod 4.294.967.296) = 4.294.967.295

Operatoren, die wraps herbeif\"uhren k\"onnen, sind:
\begin{itemize}
    \item + Addition
    \item - Subbtraktion
    \item * Multiplikation
    \item << Bitshift
\end{itemize}
\subsection{Codebeispiele}
Addition wird zwischen zwei Operanden der arithmetischen Art oder einem Zeiger auf einen Objekttyp und einem Ganzzahltyp durchgef\"uhrt. Die folgende Regel gilt nur f\"ur zwei Operanden arithmetischer Art.
\subsubsection{nichtkonformer Code}
\lstinputlisting[linerange={5-150}]{../code/aufg3/CERT-C-Coding-Standard/src/Praktikum3/int30-noncompliant.c}
In diesem Beispiel werden zwei unsigned Integers addiert, ohne zu pr\"ufen ob durch die Addition ein Wrap herbeigef\"uhrt wird.
\subsubsection{konformer Code}
\lstinputlisting[linerange={5-150}]{../code/aufg3/CERT-C-Coding-Standard/src/Praktikum3/int30-compliant.c}
\newpage
In diesem Beispiel wird die notwendige Pr\"ufung durchgef\"uhrt, bevor der Addition erfolgt. In Zeile 6 wird gepr\"uft ob die Maximalgr\"o{\ss}e eines unsigned Integers minus ui\_a kleiner ist als ui\_b. Ist ui\_b gr\"o{\ss}er als die Differenz von UINT\_MAX und ui\_a dann w\"urde ein Wrap erfolgen. Stattde{\ss}en wird hier ein Fehler geworfen.

Subtraktion wird zwischen zwei Operanden arithmetischer Art, zwei Zeigern auf qualifizierte oder unqualifizierte Versionen von kompatiblen Objekttypen oder einem Zeiger auf einen Objekttyp und einen Ganzzahltyp durchgef\"uhrt. Diese Regel gilt nur f\"ur die Subtraktion zweier Operanden arithmetischer Art.
\subsubsection{nichtkonformer Code}
Dieses nichtkonforme Codebeispiel kann zu einem unsigned integer wrap w\"ahrend der Subtraktion der Operanden \( u_ia und u_ib\) f\"uhren. Wenn dieses Verhalten nicht vorgesehen ist, kann es zu ausnutzbaren Schwachstellen kommen.
\lstinputlisting{../code/aufg3/CERT-C-Coding-Standard/src/Praktikum3/int30-2-noncompliant.c}
\subsubsection{konformer Code}
Diese konforme L\"osung f\"uhrt einen Nachbedingungstest durch, bei dem das Ergebnis der Subtraktion nicht gr\"o{\ss}er ist als der Minuend
\lstinputlisting{../code/aufg3/CERT-C-Coding-Standard/src/Praktikum3/int30-2-compliant.c}
\subsection{Auswirkung bei Nichtbeachtung}
Unsigned Integer wraps k\"onnen zu Bufferoverflows und der Ausf\"uhrung von beliebigem  Code f\"uhren.
Beispielsweise f\"uhrte ein unsigned Integer wrapping im Linux Kernel zu einem Bufferoverflow (Linux Kernel vmsplice exploit).

%\newpage
\section{MEM31-C: Free dynamically allocated memory when no longer needed}
\subsection{Beschreibung der Richtlinie}
Allokierter Speicher sollte, nachdem er nichtmehr ben\"otigt wird, freigegeben werden.
\subsection{Codebeispiele}
In diesem Beispiel wird das vom Aufruf von malloc() allokierte Objekt nicht befreit, bevor die Lebenszeit des letzten Zeigers text\_buffer, der auf dieses Objekt zeigt, endet.

\subsubsection{nichtkonformer Code}
\lstinputlisting[linerange={5-150}]{../code/aufg3/CERT-C-Coding-Standard/src/Praktikum3/mem31-noncompliant.c}
~\\
\subsubsection{konformer Code}
\lstinputlisting[linerange={5-150}]{../code/aufg3/CERT-C-Coding-Standard/src/Praktikum3/mem31-compliant.c}

\subsection{Auswirkung bei Nichtbeachtung}
Memoryleaks entstehen, der Hauptspeicher.


\newpage
\section{INT33-C: Ensure that devision and remainder operations so not in result in devide-by-zero errors}
\subsection{Beschreibung der Richtlinie}
\subsection{Codebeispiele}
Erkl\"arung und weitere Beispiele Einf\"ugen
\subsubsection{nichtkonformer Code}
\lstinputlisting[linerange={5-150}]{../code/aufg3/CERT-C-Coding-Standard/src/Praktikum3/int33-noncompliant.c}

In diesem Codebeispiel wird in Zeile 5 nur geprüft ob ein Overflow entstehen würde. Eine Division durch null kann weiterhin durchgeführt werden wenn als Argument s\_b eine null übergeben wird.

\subsubsection{konformer Code}
\lstinputlisting[linerange={5-150}]{../code/aufg3/CERT-C-Coding-Standard/src/Praktikum3/int33-compliant.c}

Bei diesem Beispiel wird in Zeile 5 geprüft ob das Argument s\_b eine Null beinhaltet. Ist dies der Fall dann wird auf den Fehler reagiert.

\subsection{Auswirkung bei Nichtbeachtung}
Divide by Zero kann zu DoS und abnormalen Programmabstürzen  führen.


\end{document}